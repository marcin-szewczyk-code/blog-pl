% File: hello-practical-pl.tex
% Author: Marcin Szewczyk
% Title: LaTeX Hello World practical example
% Created: 2026-02-20

\documentclass[a4paper,11pt]{article}

% --- Encoding & language ---
\usepackage[T1]{fontenc}
\usepackage[utf8]{inputenc}
\usepackage[polish]{babel}

% --- Layout ---
\usepackage[left=18mm,right=18mm,top=18mm,bottom=18mm]{geometry}
\usepackage{microtype}

% --- Section spacing ---
\usepackage{titlesec}
\titlespacing*{\section}{0pt}{0.8\baselineskip}{0.3\baselineskip}
\titlespacing*{\subsection}{0pt}{0.6\baselineskip}{0.2\baselineskip}

% --- Math & graphics ---
\usepackage{amsmath}
\usepackage{graphicx}
\usepackage{wrapfig}

% --- Hyperlinks ---
\usepackage{xcolor}
\usepackage[
colorlinks=true,
linkcolor=blue!50!black,
urlcolor=blue!50!black,
citecolor=blue!50!black
]{hyperref}

% --- Frame settings ---
\setlength{\fboxsep}{4pt}    % frame padding
\setlength{\fboxrule}{0.6pt} % frame thickness

% --- Metadata ---
\title{Hello World in \LaTeX: Struktura dokumentu\\i przykład (prawie) minimalny}
\author{Marcin Szewczyk}
\date{\today}

% --- Section numbering with trailing dot ---
\renewcommand{\thesection}{\arabic{section}.}
\renewcommand{\thesubsection}{\arabic{section}.\arabic{subsection}.}

\begin{document}
	
	\maketitle
	\tableofcontents
	
	\section{Wprowadzenie}
	
	\subsection{Tekst}
	
	To jest przykładowy dokument \LaTeX.
	Możemy pisać zwykły tekst oraz budować strukturę dokumentu.
	
	\subsection{Matematyka}
	
	Wzory matematyczne zapisujemy w linii
	$ e^x=\sum_{n=0}^{\infty}{x^n\over n!} $,
	lub w akapitach:
	\begin{equation}\label{eq-ex}
		e^x=\sum_{n=0}^{\infty}{x^n\over n!},
	\end{equation}
	\noindent a następnie odwołujemy się do nich~\eqref{eq-ex}.
	
	\subsection{Cytowania}
	
	Możemy cytować literaturę \cite{oppenheim2023},
	grupować cytowania \cite{oppenheim2023,stroustrup2013,grebosz2004}
	lub dodawać numery stron \cite[s.~12]{oppenheim2023}.
	
	\section{Rysunki}
	
	\begin{wrapfigure}{r}{0.42\textwidth}
		\centering
		\fbox{\includegraphics[width=0.38\textwidth]{figures/mat-gcb.pdf}}
		\caption{Przykładowy rysunek}
		\label{fig:example}
	\end{wrapfigure}
	
	Przykład wstawiania grafiki.
	Można odwoływać się do rysunku (rys.~\ref{fig:example}).
	
	\section*{Bibliografia}
	
	\begin{thebibliography}{99}
		
		\bibitem{oppenheim2023}
		T.~Oetiker, H.~Partl, I.~Hyna, E.~Schlegl,
		\textit{The Not So Short Introduction to \LaTeX2e}, 2023.
		
		\bibitem{stroustrup2013}
		B.~Stroustrup,
		\textit{The C++ Programming Language},
		4th ed., Addison-Wesley, 2013.
		
		\bibitem{grebosz2004}
		J.~Grębosz,
		\textit{Symfonia C++ Standard. Programowanie w języku C++},
		Edition 2004.
		
	\end{thebibliography}
	
\end{document}
